\documentclass[12pt, oneside]{article}

% Language setting
% Replace `english' with e.g. `spanish' to change the document language
\usepackage[english]{babel}

% Set page size and margins
% Replace `letterpaper' with `a4paper' for UK/EU standard size
\usepackage[letterpaper,top=2cm,bottom=2cm,left=3cm,right=3cm,marginparwidth=1.75cm]{geometry}

% Useful packages
\usepackage{amsmath}
\usepackage{graphicx}
\usepackage[colorlinks=true, allcolors=blue]{hyperref}
\usepackage{graphicx} % Required for inserting images
\usepackage{cite}
\usepackage{amsmath,amssymb,amsfonts}
\usepackage{algorithmic}
\usepackage{graphicx}
\usepackage{textcomp}
\usepackage{xcolor}
\usepackage{csquotes}
\usepackage{placeins}
\usepackage{etoolbox}
\usepackage{IEEEtrantools}

\begin{document}

\begin{titlepage}
    \begin{center}
        \vspace*{1cm}
        {\huges
        \center{\huge{\textbf{Advanced Control and Dynamics}}}}
         \\
         \vspace{0.3cm}
         \large{Coursework Report}
         \vspace{0.5cm}
        \\
        {\large By}
        \\
        \vspace{0.5cm}
        \textbf{Runze Yuan}
        \\
        \vspace{0.5cm}
        Student Number: 22071714
   		\vspace{1.5cm}
        \\
        \vspace{0.25cm}
       \includegraphics[scale=0.6]{logos/bristolcrest_colour.pdf}
        \hspace{5mm}
        \includegraphics[scale=0.35]{logos/UWE_insignia.png}

        \vspace{10mm}
        {\large Department of Engineering Mathematics\\
        \textsc{University of Bristol}}
        \\
        \&
        \\
        {\large Department of Engineering Design and Mathematics\\
        \textsc{University of the West of England}}\\

        \vspace{0.8cm}
 
        \vspace{0.8cm}
        \today
        
    \end{center}
    
\end{titlepage}

\tableofcontents
\pagebreak

\section{Practical Plant Definition}

% http://sim.okawa-denshi.jp/en/CRCRkeisan.htm


\textbf{Question:} 
\begin{quote}
Define a practical engineering plant which would feature similar dynamical behaviour to the theoretical dynamics given in the plant description below. Briefly describe the operation of the plant.
\end{quote}

Theoretical transfer function of the plant:

\begin{equation}
    \frac{Y(s)}{U(s)} = g_{p}(s) = \frac{1}{s^{2}+0.6s+4}
\end{equation}


\textbf{Answer:}
% https://electronics.stackexchange.com/questions/152159/deriving-2nd-order-passive-low-pass-filter-cutoff-frequency
\vspace{0.5cm}

A common example which has the given transfer function is a RLC circuit.





\section{Control System Block Diagrams}
\section{Plant Analyse}
\section{State Feedback Controller Design}
\section{Observer Design}
\section{Performance Simulation}
\section{Digital Controller Implementation}
\section{References}

\end{document}